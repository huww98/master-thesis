\documentclass{ctexart}

\title{基于可微分渲染的人脸3D重建}
\author{胡玮文}

\begin{document}

\maketitle

\tableofcontents

\section{绪论}

\section{相关研究综述}

\section{适应未知背景的无偏可微分渲染技术}

\subsection{问题定义}

可微分渲染旨在从参数化的3D模型,如三角形面片,纹理贴图等,渲染得到2D图片,同时计算该渲染结果关于其输入参数的梯度,以期利用基于梯度的方法优化输入参数,从而改善渲染效果。

然而现有方法存在一些不足:
\begin{enumerate}
    \item 现有可微分渲染技术大多是对整张图片估计梯度,即不区分前景和背景部分。然而,在基于自然环境照片的人脸重建任务中,我们通常只有前景(即人脸)的3D模型,而没有背景的模型。若只使用人脸模型拟合整张图片则无法得到合理的结果,在缺乏关于照片背景的先验知识时也难以对照片背景进行显式建模。
    \item 现有的一些梯度估计方法是有偏的,即该梯度无法指引模型收敛到局部最优点。另一些无偏的方法则有一些其他缺陷,有些计算量大、梯度噪声大,有些会在渲染结果中引入少量模糊等。这些缺陷会导致模型收敛的速度和精度下降,甚至无法收敛到较合理的解。
\end{enumerate}

\subsection{本文方法}

\paragraph{收缩\&扩展约束}(创新点)

\paragraph{基于纹理坐标的无偏几何梯度估计}(预估创新点,尚未实验)

\subsection{实验结果}

\subsection{讨论}

\section{基于可微分渲染的人脸3D重建}

\subsection{基于单张自然环境照片的人脸重建}

\subsection{基于多视角实验室照片的人脸重建}

\subsubsection{问题定义}

输入多个相机在(几乎)同一时刻拍摄的肖像照片。相机的参数和环境光照可以提前标定。本文的目标是静态模型重建,因此无须考虑采集连续动态图像的问题。通过以很小的间隔(数毫秒)触发多盏闪光灯,也能在单次采集中获得不同照明条件的数据。

输出被拍摄对象尽可能精确的3D模型,并能支持任意视角,任意光照条件下重新渲染。

本文的贡献主要在于采集系统的搭建方案,以及结合我们特定情况的算法复现和改进。

\subsubsection{相机固定}

铝型材支架设计

\subsubsection{多相机内外参联合标定}

\paragraph{二维码识别}尺度、对比度无关的二维码定位算法(创新点?不知道够不够格)

\paragraph{角点精确定位}二维多项式函数的拟合与鞍点查找

\paragraph{多相机内参标定与外参传递}

\paragraph{集束调整}

\paragraph{标定结果和误差分析}

\subsubsection{被动相机同步}

该装置无须独立供电。它能以很高的精度同时触发多台单反相机的对焦和快门,从而实现人脸多视角数据的捕获。

\paragraph{单反相机快门触发原理}

\paragraph{同步装置硬件设计}

\paragraph{同步精度测试}

\subsubsection{主动相机/闪光灯同步}

主动相机同步装置有单片机等实时控制系统控制,能独立控制每台相机、闪光灯触发延迟,最多24个通道。可用于快速抓拍不同闪光灯照明下的照片

\paragraph{主动同步装置硬件设计}

\paragraph{主动同步装置软件设计}

\paragraph{滚动快门原理}

\paragraph{闪光灯触发延迟快速标定方法}

\subsubsection{基于反射球的光源标定和HDRI合成}

\subsubsection{基于可微分渲染的3D重建}

\paragraph{3D模型初始化}

\paragraph{基于物理的可微分逆渲染}

\section{结论与展望}

\section*{参考文献}

\section*{致谢}

\end{document}
