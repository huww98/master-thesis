\documentclass{ctexart}

\usepackage{hyperref}
\usepackage{amssymb}
\usepackage{amsmath}

\title{基于可微分渲染的人脸3D重建}
\author{胡玮文}

\begin{document}

\maketitle

\tableofcontents

\section{绪论}

\section{相关研究综述}

\section{适应未知背景的无偏可微分渲染技术}

\subsection{问题定义}

可微分渲染旨在从参数化的3D模型,如三角形面片,纹理贴图等,渲染得到2D图片,同时计算该渲染结果关于其输入参数的梯度,以期利用基于梯度的方法优化输入参数,从而改善渲染效果。

然而现有方法存在一些不足:
\begin{enumerate}
    \item 现有可微分渲染技术大多是对整张图片估计梯度,即不区分前景和背景部分。然而,在基于自然环境照片的人脸重建任务中,我们通常只有前景(即人脸)的3D模型,而没有背景的模型。若只使用人脸模型拟合整张图片则无法得到合理的结果,在缺乏关于照片背景的先验知识时也难以对照片背景进行显式建模。
    \item 现有的一些梯度估计方法是有偏的,即该梯度无法指引模型收敛到局部最优点。另一些无偏的方法则有一些其他缺陷,有些计算量大、梯度噪声大,有些会在渲染结果中引入少量模糊等。这些缺陷会导致模型收敛的速度和精度下降,甚至无法收敛到较合理的解。
\end{enumerate}

\subsection{本文方法}

\paragraph{收缩\&扩展约束}(创新点)

\paragraph{基于纹理坐标的无偏几何梯度估计}(预估创新点,尚未实验)

\subsection{实验结果}

\subsection{讨论}

\section{基于可微分渲染的人脸3D重建}

\subsection{基于单张自然环境照片的人脸重建}

\subsection{基于多视角实验室照片的人脸重建}

\subsubsection{问题定义}

输入多个相机在(几乎)同一时刻拍摄的肖像照片。相机的参数和环境光照可以提前标定。本文的目标是静态模型重建,因此无须考虑采集连续动态图像的问题。通过以很小的间隔(数毫秒)触发多盏闪光灯,也能在单次采集中获得不同照明条件的数据。

输出被拍摄对象尽可能精确的3D模型,并能支持任意视角,任意光照条件下重新渲染。

本文的贡献主要在于采集系统的搭建方案,以及结合我们特定情况的算法复现和改进。

\subsubsection{相机固定}

铝型材支架设计

\subsubsection{多相机内外参联合标定}

为了建立三维物体与二维照片之间的准确对应关系,需要对相机的内参和外参进行标定。
即准确测量采集过程中用到的每一台相机的内参和外参。
其中,内参包括相机的焦距、光心坐标、畸变参数等,
外参则包括不同相机之间的相对位置和姿态。

本文选用的相机模型是针孔相机模型,这也是在实验中采用的相机所遵循的模型。
由于本实验中相机畸变较小,简单起见,本文选用了OpenCV中默认的径向和切向相机畸变模型\cite{?}。
更正式地,假设共有N个相机,对于第$i$个相机($i=1,2,\cdots,N$),
内参标定的目标是求解每个相机的
焦距$f_x^{(i)},f_y^{(i)}$、
光心坐标$c_x^{(i)},c_y^{(i)}$、
畸变参数$k_1^{(i)},k_2^{(i)},k_3^{(i)},p_1^{(i)},p_2^{(i)}$;
外参标定的目标则是求解每个相机在世界坐标系下的
位置$\mathbf{t}^{(i)}=\left(x^{(i)},y^{(i)},z^{(i)}\right)$
和姿态$\mathbf{r}^{(i)}$。
其中$\mathbf{r}\in \mathbb{R}^3$为表示旋转的罗德里格斯向量\cite{?}。
世界坐标系的选取是任意的,因此在相机标定阶段,不失一般性地,我们选择第一个相机作为世界坐标系。
即令$\mathbf{t}^{(1)}=\mathbf{0}$,$\mathbf{r}^{(1)}=\mathbf{0}$。
综上,在标定过程中共需要求解$N\times 15 - 6$个参数。

在该模型下,对于任意在世界坐标系下的点$\mathbf{X}=\left[X,Y,Z\right]^\mathsf{T}$,其在第$i$个相机的成像平面的投影点$\mathbf{x}^{(i)}$可以通过如下方式计算:
首先,将$\mathbf{X}$从世界坐标系转换到第$i$个相机的坐标系,即
\begin{align}
    \theta &= \left\|r\right\|_2 \\
    \hat{\mathbf{r}} &= \mathbf{r}/ \theta \\
    \mathbf{R} &= \cos(\theta) I + (1- \cos{\theta} ) \hat{\mathbf{r}} \hat{\mathbf{r}}^\mathsf{T} + \sin(\theta) \begin{bmatrix}
         0   & -\hat{\mathbf{r}}_z & \hat{\mathbf{r}}_y \\
         \hat{\mathbf{r}}_z & 0    & -\hat{\mathbf{r}}_x \\
        -\hat{\mathbf{r}}_y &  \hat{\mathbf{r}}_x & 0
    \end{bmatrix} \\
    \mathbf{X}' &= \mathbf{R} \mathbf{X} + \mathbf{t}
\end{align}
然后,将$\mathbf{X}'$投影到第$i$个相机的成像平面,即
\begin{align}
    \mathbf{X}'' &= \begin{bmatrix}
        f_x & 0 & c_x \\
        0 & f_y & c_y \\
        0 & 0 & 1
    \end{bmatrix} \mathbf{X}' \\
    \mathbf{x}'' &= \begin{bmatrix}
        \mathbf{X}''_x / \mathbf{X}''_z \\
        \mathbf{X}''_y / \mathbf{X}''_z
    \end{bmatrix}
\end{align}
最后,对$\mathbf{x}''$应用镜头畸变,即
\begin{align}
    \mathbf{x}' &= \left(1 + k_1 r^2 + k_2 r^4 + k_3 r^6\right) \mathbf{x}'' \\
    \mathbf{x} &= \mathbf{x}' + \begin{bmatrix}
        2 p_1 x'_x x'_y + p_2 \left(r^2 + 2 (x'_x)^2\right) \\
        p_1 \left(r^2 + 2 (x'_y)^2\right) + 2 p_2 x'_x x'_y
    \end{bmatrix}
\end{align}


\paragraph{二维码识别}尺度、对比度无关的二维码定位算法(创新点?不知道够不够格)

\paragraph{角点精确定位}二维多项式函数的拟合与鞍点查找

\paragraph{多相机内参标定与外参传递}

\paragraph{集束调整}

\paragraph{标定结果和误差分析}

\subsubsection{被动相机同步}

该装置无须独立供电。它能以很高的精度同时触发多台单反相机的对焦和快门,从而实现人脸多视角数据的捕获。

\paragraph{单反相机快门触发原理}

\paragraph{同步装置硬件设计}

\paragraph{同步精度测试}

\subsubsection{主动相机/闪光灯同步}

主动相机同步装置有单片机等实时控制系统控制,能独立控制每台相机、闪光灯触发延迟,最多24个通道。可用于快速抓拍不同闪光灯照明下的照片

\paragraph{主动同步装置硬件设计}

\paragraph{主动同步装置软件设计}

\paragraph{滚动快门原理}

\paragraph{闪光灯触发延迟快速标定方法}

\subsubsection{基于反射球的光源标定和HDRI合成}

\subsubsection{基于可微分渲染的3D重建}

\paragraph{3D模型初始化}

\paragraph{基于物理的可微分逆渲染}

\section{结论与展望}

\section*{参考文献}

\section*{致谢}

\end{document}
