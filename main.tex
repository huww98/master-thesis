\documentclass{scutmaster}

\usepackage{pgf}
\usepackage{import}
\usepackage{graphicx}
\usepackage{booktabs}
\usepackage{multirow}
\usepackage{siunitx}
\usepackage{xcolor}
\usepackage{subcaption}
\usepackage[author={胡玮文}]{pdfcomment}
\usepackage{tikz}
\usepackage{wrapfig}
\usetikzlibrary{backgrounds,intersections,calc,positioning,fit,shapes.geometric}
\usepackage{hyperref}

\DeclareMathOperator*{\argmax}{arg\,max}
\DeclareMathOperator*{\argmin}{arg\,min}
\DeclareMathOperator{\sg}{sg}

\newcommand{\TODO}[1]{\textcolor{red}{TODO: #1}\GenericWarning{}{LaTeX Warning: TODO: #1}}

\title{基于可微分渲染的3D人脸重建方法研究}
\titleEN{3D Face Reconstruction Based on Differentiable~Rendering}
\date{\zhtoday}
\classificationnumber{TP37}

\author{胡玮文}
\authorEN{Weiwen Hu}
\studentnumber{202045611}
\phone{17701952145}
\email{sehuww@mail.scut.edu.cn}
\address{江西省南昌市青山湖区江大南路139号荣昌小区16栋(330029)}

\degree{电子信息硕士(软件工程)}{电子信息硕士}
\major{软件工程}{数字人}

\supervisor{杜卿}{副教授}
\supervisorEN{Assoc. Prof.}{Qing Du}

\school{软件学院}

\input{build/git_description}
\hypersetup{
    pdfinfo={
        Version=\gitdescription
    }
}

\begin{document}

\maketitle
\hideinblind{
    \maketitleEN
    \nominationpage
    \declareoforiginality
}

\frontmatter
\chapter{摘\texorpdfstring{\quad}{}要}

3D人脸重建技术旨在从非受限环境拍摄的照片中恢复出人脸的3D模型,
这些模型已被广泛应用于人脸识别、人脸表情捕捉、人脸跟踪、人脸动画合成等任务中。
然而,基于少量非受限环境照片重建的模型还难以应用于需要直接渲染图像的游戏、影视等场景中,
具体来说,现有的基于非受限环境照片的重建方法还存在如下问题:
现有方法未能充分利用照片中的边缘信息,它们依赖2D人脸关键点识别、多目立体等方法以重建人脸的几何形状,这会引入额外的复杂性和累积误差。

为解决非受限环境照片中边缘信息难以利用的问题,
本文提出了一种基于可微分渲染的3D人脸重建算法。
本文从理论上分析了无法对背景建模时的可见性梯度计算问题,
提出了一种面积归一化的像素损失函数。
本文分析了该损失函数的作用机理,并与现有的通用可微分渲染研究中的可见性梯度计算方法相结合,高效实现了该损失函数。

然而人脸只是整个人的一部分,当人脸模型从整个人上裁剪下来用于重建时,在人工裁剪的模型边缘会产生异常梯度。
为解决该问题,
本文进一步提出了一种基于SDF贴图的方法对所有渲染图像中的边缘进行分类,从而消除这些异常梯度,
最终使人脸3D模型能更精确地对齐到照片中的边缘,从而提高了重建的准确性和实用性。

基于本文提出的这些方法,本文实现了一个基于逆渲染的完整自动化人脸重建流程。
该流程利用现有神经网络方法进行初始化,并结合传统方法重建人脸纹理。
其重建的人脸可在一定的视角、光照、表情变化下较为逼真的重新渲染。
本文展示和评估了其重建效果。

\keyword{关键词:} 可微分渲染;逆渲染;人脸重建

\chapter{Abstract}

3D face reconstruction aims to recover 3D models of faces from a small number of unconstrained photos,
which have been widely applied to face recognition, face expression capturing, face tracking, face animation synthesis, etc.
However, these models are still difficult to apply to scenes that require direct image rendering, such as games and movies.
Specifically, the existing reconstruction methods still have the following problem:
The existing methods do not fully utilize the edge information in the photos,
instead, they rely on 2D face landmark detection, multi-view stereo, etc.\ to reconstruct the geometric shape of the face,
which introduces additional complexity and cumulative error.

To address the difficulty of utilizing edge information in unconstrained photos,
this paper proposes a 3D face reconstruction algorithm based on differentiable rendering.
This paper analyzes the problem of computing visibility gradients in situations where the background cannot be modeled,
and proposes an area-normalized pixel loss function.
This paper analyzes the mechanism of this loss function,
and combines it with the existing methods for computing visibility gradients in general differentiable rendering to implement it efficiently.

However, the face model is only part of the whole human model.
When the face model is cropped for reconstruction,
there will be abnormal gradients at the artificial cut edges.
To overcome this issue, this paper further proposes a method based on SDF maps to classify all edges in the rendered images,
which eliminates these abnormal gradients.
The proposed method enhances the accuracy and practicality of the reconstruction
by enabling the face 3D model to align more precisely with the edges in the photo.

Based on the proposed methods, This paper implements a complete automatic face reconstruction pipeline based on inverse rendering.
It employs neural network methods for initialization and combines traditional methods to reconstruct the face texture.
The resulting reconstructed face can be realistically re-rendered under certain views, lighting, and expression changes.
This paper presents and evaluates the reconstruction results.

\keyword{Keywords:} Differentiable rendering; Face reconstruction; Face scanning

\tableofcontents

\listoffigures

\listoftables

\mainmatter
\chapter{绪论}
\label{chap:intro}

\section{研究背景}

\section{本文研究内容及贡献}

\section{本文组织结构}

本文正文共分为六个章节,各章节的主要内容如下:

\newcommand*{\chapref}[1]{\hyperref[{#1}]{第\ref*{#1}章~\nameref*{#1}}}

\chapref{chap:intro}。
本章介绍了3D人脸重建、以及可微分渲染的背景、关联及意义,
并总结了本文主要贡献及各章节的内容安排。

\chapref{chap:related_work}。
本章对3D人脸重建、可微分渲染相关方向的研究和实践进行了综述,
其中包括\TODO{}。
此外,还介绍了一些可用于3D人脸渲染的计算机图形学的基础知识。

\chapref{chap:method}。
本章介绍了本文提出的一种对现有可微分渲染的改进方法。
该方法能增强可微分渲染技术在难以建模背景的图像拟合任务中的适应能力。
其理论构建在修改的损失函数之上,并通过收缩、扩张两项梯度项来实现。
本章也通过实验验证了该方法的有效性。

\chapref{chap:recon}。
利用上一章提出的方法,本章介绍了一个从单张自然环境照片重建3D人脸的实现。
同时该实现有机结合了神经网络、可微分渲染和一些传统算法,
达成了快速高效,细节丰富且全自动的3D人脸模型重建。
本章也与其他相关工作进行了比较。

\chapref{chap:platform}。
为了更专业地采集基于物理的人脸渲染所需的材质数据,实现更高精度,更逼真的3D人脸渲染效果,
本章介绍了一个多视角人脸重建实验平台。
该平台能够实现高分辨率人脸照片的便捷多视角采集,
并为后续的数据收集整理,相机、光源标定等重建所必备的步骤提供了软件支持。
本章将介绍该平台的软硬件设计思路、实现细节、使用方法以及其可行性初步验证的结果。

\chapref{chap:conclusion}。
基于本文完成的工作结果对本文研究进行总结,并分析当前工作存在的不足和对未来工作的展望。


\chapter{相关研究综述}
\label{chap:related_work}

\section{多视角3D重建}

\section{高效3D人脸重建}

\section{可微分渲染}

\section{高质量人脸3D渲染}

\paragraph{基于物理的渲染}

\paragraph{分离求和近似}

\paragraph{次表面散射}


\chapter{适应未知背景的可微分渲染方法}
\label{chap:method}

在将可微分渲染方法应用到3D人脸重建任务时,现有方法均未能充分利用可见性梯度。
而该梯度正是所有梯度中的主要分量,对模型与照片的精确对齐有着重要作用。
通常来说,计算正确的可见性梯度需要同时具有前景和背景的模型,而在自然环境照片中,由于背景多样,很难得到显式的背景模型。
本章主要针对该问题,提出一种适应未知背景的可微分渲染方法,能够利用可见性梯度来优化模型与照片的对齐。

\section{问题定义}

可微分渲染旨在从参数化的3D模型,如三角形面片,纹理贴图等,渲染得到2D图片,同时计算该渲染结果关于其输入参数的梯度。
然后可以优化渲染结果与照片间的误差,以期利用基于梯度的方法优化输入参数,从而改善渲染效果,使之更加接近现实。
其在3D人脸重建的相关任务中已有较广泛的应用。

然而现有方法存在一些不足:
现有可微分渲染技术大多是对整张图片估计梯度,即不区分前景和背景部分。
但是,在基于自然环境照片的人脸重建任务中,我们通常只有前景(即人脸)的3D模型,而没有背景的模型。
此时,现有方法选择忽略模型间相互遮挡产生的可见性梯度,而这会造成模型与照片的对齐不准确,如图\ref{fig:problem_a}所示。
另一方面,若对背景进行很粗糙的建模,例如假设为全黑,则错误的梯度会导致模型无法正确收敛,如图\ref{fig:problem_b}所示。

\begin{figure}
\centering
\begin{subfigure}[t]{0.45\textwidth}
    \centering
    \includegraphics[width=\textwidth]{figures/black-bg_no-aa}
    \caption{忽略可见性梯度,模型与照片未准确对齐}
    \label{fig:problem_a}
\end{subfigure}
\begin{subfigure}[t]{0.45\textwidth}
    \centering
    \includegraphics[width=\textwidth]{figures/black-bg}
    \caption{使用全黑代替背景模型,可见性梯度错误}
    \label{fig:problem_b}
\end{subfigure}
\caption[未知背景条件下可微分渲染优化结果]{
    未知背景条件下可微分渲染优化结果。
    示意图为目标照片和渲染结果以1:1 alpha混合后,再进行gammar校正得到。
}
\end{figure}

为绕过上述问题,现有方法通常:
使用多目立体等其他手段事先确定高精度的人脸几何形状,并在逆渲染优化过程中使几何形状保持不变\citep{RiviereGBGB20};
忽略可见性梯度,使用2D人脸关键点\citep{deep3d}、图像分割\citep{nvdiffrec}结果等辅助监督信息来补足这部分缺失的梯度。
但是,如果能够直接利用可见性梯度,可大大简化算法的流程,同时也能避免在前序步骤中引入额外的误差。

本节将设计一个简单的玩具实验来更直观地说明上述问题。
如图\ref{fig:problem}所示,我们使用一个简单的白色平面作为前景模型,用其拟合一个白色青色对半分割的场景。
该模型的参数是其横向的平移量,用平面边缘与横坐标轴的交点$a_x$表示。
该平面的边缘稍微倾斜,以突出可见性梯度的变化。
该平面的着色不受参数影响,始终固定为白色。
该例子中使用的损失函数为L1误差,即:
\begin{equation}
\mathcal{L} = \left\| \hat{\mathcal{I}} - \mathcal{I} \right\|_1
\text{。}
\end{equation}

\begin{figure}
\centering
\import{build/figures}{problem.pgf}
\caption[在未知背景时计算可见性梯度困难]{
    在未知背景时计算可见性梯度困难。
    拟合目标和渲染结果左侧白色为前景,右侧为背景。
    前景模型为以红线为边界的白色平面。
    横坐标为前景模型的平移量。
    a)离散采样,不计算可见性梯度;
    b)使用不准确的黑色背景模型计算可见性梯度,模型无法收敛到期望位置;
    c)理想情况下,在完全已知背景时的可见性梯度,模型能准确收敛至梯度为0的点。
}
\label{fig:problem}
\end{figure}

其中,a)为离散采样的方式,也是目前大多人脸重建中使用的方法。
其渲染流程为:在每个像素的中点处采样,若其位于前景模型的覆盖范围内,则渲染为白色,否则渲染为背景。
由于采样过程是离散的,且$a_x$的改变并不会影响采样的模式,因此,渲染结果$\hat{\mathcal{I}}$关于$a_x$的梯度为0,其损失函数为阶梯函数。
即该损失函数完全无法以基于梯度的方式指导模型拟合。

另一方面,在没有准确背景模型的前提向,可见性梯度则难以被利用。
例如,在b)中,假设我们并不知道该场景的背景,因此直接假设其为全黑。
但在本例中这个假设显然是不准确的。
在此前提下,从图中可以观察到其损失函数呈递减的趋势,并未在期望的位置上形成极值。
因此,该梯度也无法引导模型收敛到期望的位置。

作为参考,c)展示了理想中,完全已知背景的情况下,可见性梯度的作用。
从图中可见,其损失函数在期望的位置上有一个明显的极值,且其周围的梯度将能很好地指导优化器,使模型收敛到该位置。

然而,在实际人脸重建的任务中,特别是从自然环境照片的重建中,背景可能是很复杂且难以建模的。
本章将探讨如何在这种情况下,如何利用可见性梯度以指导前景模型与目标照片对齐。

\section{本文方法}

收缩\&扩展约束(创新点)

\section{实验结果}

\section{讨论}

L1损失函数的必要性


\chapter{基于单张自然环境照片的人脸重建}
\label{chap:recon}

\section{总体目标}

\section{基于神经网络回归的模型初始化}

\section{模型裁剪边缘的梯度处理}
可微分渲染在应用到人脸3D重建时有一个特殊的问题:人是一个很大的物体,而人脸重建的目标通常仅仅是人脸的这一小块区域。
所以通常在人脸重建中用于逆渲染的模型都是裁剪过的,只包含人脸的部分,因此具有一些人为制造的边缘。
这些边缘在照片中并不存在,但本文提出的方法却也会在这些边缘产生相关梯度,导致模型过度扩展,影响最终收敛的效果。对此,本文提出一种简单的基于SDF贴图的方法来区分人为制造的和真正需要优化的边缘。

具体地,本文定义一种SDF贴图。
对于该贴图中的每个像素,其值为一个单通道浮点数$d$,表示该像素点到最近的人为制造边缘的距离,在模型内部的为正,模型外部的为负,
如图\ref{fig:sdf}所示。
nvdiffrast抗锯齿模块工作时将以一对相邻的像素点为单位(如图\ref{fig:aa}),其中一个像素在模型内部,另一个像素在模型外部。
对于处于模型内部的像素,可从SDF贴图中采样其对应的$d$值,并计算$\frac{\partial d}{\partial \mathbf{x}}$,其中$\mathbf{x}$为对应外部像素相对该内部像素的坐标偏移量。
也即对3D模型的表面在该内部像素点的位置使用一个平面进行一阶近似,并求外部像素在该近似平面上的位置,以估计外部像素是否在人为制造边缘之外。
此外,为避免过大梯度是位置估计不准确,本文限制了$\frac{\partial d}{\partial \mathbf{x}}$的最大值。正式地,对于一对分别处于模型内外的像素点,其处于人为制造的边缘当且仅当:
\begin{equation}
    d + \min\left(\frac{\partial d}{\partial \mathbf{x}}, \delta\right) < 0
    \text{,}
\end{equation}
此时应该忽略其产生的可见性梯度。

另一种可能的方案是屏蔽位于模型表面特定区域的像素上的可见性梯度,例如屏蔽特定三角形,到边缘距离小于某个阈值的区域,手动指定遮罩等。
相比于这些方法,本方法具有如下优势:
\begin{enumerate}
\item 渲染尺度无关:同样的模型区域在不同尺度下渲染呈现的尺度不同,因此屏蔽区域的选择可能需要随着尺度变化而改变。
而本方法融入$d$了对像素坐标的梯度信息,因此不受尺度的影响。
\item 有向性:即使是同一个像素,在其不同方向上也可能分别靠近不同种类的边缘。
本方法通过考虑$d$的梯度方向,可以区分这些不同的边缘。
\end{enumerate}

\section{照片到纹理空间信息迁移}

\section{基于可微分渲染的3D重建}

\section{实验结果}

\section{局限性}


{
\backmatter
\chapter{结论与展望}
\label{chap:conclusion}

本文研究了高效3D人脸重建的问题,
提出了一种基于可微分渲染的高效3D人脸重建算法。
在此之上,本文最终实现了一个完整的人脸重建流程,并取得了较为令人满意的效果。
本文的主要贡献可概括如下:
\begin{enumerate}
\item 本文提出并实现了一种适应未知背景的可微分逆渲染方法,
其中包括了一种面积归一化的像素损失函数,及其基于nvdiffrast的高效实现。
该方法充分利用可见性梯度,能有效将人脸模型对齐到照片中的边缘,并有应用到其他领域的潜力。
\item 本文实现了一个完整基于单张非受限环境照片的人脸重建流程,
在上一点的基础上,本文消除了其应用于人脸模型时的异常梯度。
本文还结合了传统算法以重建纹理细节,实现了鲁棒且自动的3D人脸重建,
并展示了其重建精度和重新渲染的效果。
\end{enumerate}

虽然在限定的范围内,本文取得了较为令人满意的重新渲染效果,
然而还需要看到,相比于业界的先进方案,本文所述的内容在3D人脸重建的应用领域内尚处于蹒跚学步的阶段,
距离真正影视级的工业应用依然较远,还有大量的问题需要解决:
\begin{enumerate}
\item 如何利用高质量的影视级人脸模型来补充高效重建算法的训练数据,从而改善其细节渲染效果,这依然是前沿的研究领域。
新兴的扩散模型与3D人脸模型的结合可能是一个有前景的研究方向。
\item 尽管已经取得了不错的几何精度,本文尚只利用了较为简单的3DMM模型作为先验知识。
本文偶尔会重建出明显不合常理的人脸形状,
更为复杂的以非线性模型建模的人脸先验或能在与逆渲染方法相结合时展现更好的效果。
这也有待进一步探索。
\item 本文所实现的基于单张非受限环境照片的人脸重建流程仍是基于非常基础的渲染模型,
该模型并不能建模镜面反射,次表面散射等较为复杂的光线传播现象。
这导致本文重建的模型仅能在与照片中较为相似的环境光照和视角下重新渲染。
若要实现在影视中的大范围光照和视角变化,则需要有更多参数的更复杂的模型,也意味着需要更高质量的先验知识来重现一些微妙的效果。
\end{enumerate}

展望未来,为更好地建模人脸的先验知识,神经网络等人工智能技术将必不可少。
由于可微分渲染技术的发展应用,3D人脸重建和计算机图形学的关系也愈发密切。
然而计算机图形学独立于人工智能技术已经有了很长的历史,自身已经有了丰厚的技术积累,
笔者的硕士生涯太短,尚未能在这方面有足够的理解和实践。
未来这两个方向的研究者们或许应该加强交流合作,以进一步提升3D人脸模型的制作效率,降低成本。

% 2001年,长篇小说《刀剑神域》面世,其中展示了作者对2022年的科幻设想。
% 故事中人们使用名为“NERvGear”的小型民用终端即可以自身真实形象进入VR游戏中。
% 《刀剑神域》及其同名动画给笔者留下了深刻的印象,
% 如今2022年已过,希望我们也能早日在现实中见证这样的科技。

\bibliography{main}

\hideinblind{
\chapter{致谢}

研究生三年的时光转瞬即逝。
这三年来,我虽有遗憾,但更多的是收获和成长,并度过了一段充实的学生生涯。
相比三年前,我在知识的深度和广度上均有提升,
我深知,这是在导师的指导和同学,以及其他一些人的帮助下才能实现的,我向他们表示感谢。
我首先要感谢我的导师杜卿老师,她温柔耐心、循循善诱,她也对本文的写作给予了很多的建议。
我还要感谢谭明奎教授,他在我的研究方向尚不明朗时为我提供了很多建议和帮助。
谭教授也为我们创造了舒适的实验室环境,并提供了高性能的机器学习计算资源。
特别要感谢CVTE和王乃洲博士,本文的工作也是华南理工大学与CVTE合作项目的成果,
CVTE为本文的所有实验提供了经济支持。
同时,我要感谢我的同学们,与他们相处我感到很快乐,他们也给予了我很多的帮助。
最后,感谢强盛的祖国,和平的时代,富饶的社会,以及无数为之而奋斗的先辈们,所有人的努力造就了今天令人安心的科研环境。
}% end hideinblind

\chapter{攻读博士/硕士学位期间取得的研究成果}

一、已发表(包括已接受待发表)的论文,以及已投稿、或已成文打算投稿、或拟成文投稿的论文情况:

\noindent\begin{tabularx}{\textwidth}{| p{0.5cm} | X | p{1.8cm} | p{1.7cm} | p{1.8cm} | p{1.8cm} |}
\hline
\textbf{序号} & \textbf{发表或投稿刊物/会议名称} & \textbf{作者} & \textbf{发表年份} & \textbf{与学位论文哪一部分相关} & \textbf{被索引收录情况} \\
\hline
 & & & & & \\
\hline
\end{tabularx}

\vskip 0.5cm

二、与学位内容相关的其它成果(包括专利、著作、获奖项目等):

}% end backmatter

\end{document}
