\chapter{适应未知背景的可微分渲染方法}
\label{chap:method}

在将可微分渲染方法应用到3D人脸重建任务时,现有方法均未能充分利用可见性梯度。
而该梯度正是所有梯度中的主要分量,对模型与照片的精确对齐有着重要作用。
通常来说,计算正确的可见性梯度需要同时具有前景和背景的模型,而在自然环境照片中,由于背景多样,很难得到显式的背景模型。
本章主要针对该问题,提出一种适应未知背景的可微分渲染方法,能够利用可见性梯度来优化模型与照片的对齐。

\section{问题定义}

可微分渲染旨在从参数化的3D模型,如三角形面片,纹理贴图等,渲染得到2D图片,同时计算该渲染结果关于其输入参数的梯度。
然后可以优化渲染结果与照片间的误差,以期利用基于梯度的方法优化输入参数,从而改善渲染效果,使之更加接近现实。
其在3D人脸重建的相关任务中已有较广泛的应用。

然而现有方法存在一些不足:
现有可微分渲染技术大多是对整张图片估计梯度,即不区分前景和背景部分。
但是,在基于自然环境照片的人脸重建任务中,我们通常只有前景(即人脸)的3D模型,而没有背景的模型。
此时,现有方法选择忽略模型间相互遮挡产生的可见性梯度,而这会造成模型与照片的对齐不准确,如图\ref{fig:problem_a}所示。
另一方面,若对背景进行很粗糙的建模,例如假设为全黑,则错误的梯度会导致模型无法正确收敛,如图\ref{fig:problem_b}所示。

\begin{figure}
\centering
\begin{subfigure}[t]{0.49\textwidth}
    \centering
    \includegraphics[width=\textwidth]{figures/black-bg_no-aa}
    \caption{忽略可见性梯度,模型与照片未准确对齐}
    \label{fig:problem_a}
\end{subfigure}
\begin{subfigure}[t]{0.49\textwidth}
    \centering
    \includegraphics[width=\textwidth]{figures/black-bg}
    \caption{使用全黑代替背景模型,可见性梯度错误}
    \label{fig:problem_b}
\end{subfigure}
\caption[未知背景条件下可微分渲染优化结果]{
    未知背景条件下可微分渲染优化结果。
    示意图为目标照片和渲染结果以1:1 alpha混合后,再进行gammar校正得到。
}
\end{figure}

为绕过上述问题,现有方法通常:
使用多目立体等其他手段事先确定高精度的人脸几何形状,并在逆渲染优化过程中使几何形状保持不变\citep{RiviereGBGB20};
忽略可见性梯度,使用2D人脸关键点\citep{deep3d}、图像分割\citep{nvdiffrec}结果等辅助监督信息来补足这部分缺失的梯度。
但是,如果能够直接利用可见性梯度,可大大简化算法的流程,同时也能避免在前序步骤中引入额外的误差。

本节将设计一个简单的玩具实验来更直观地说明上述问题。

\begin{figure}
\centering
\import{build/figures}{no_aa.pgf}
\caption{test}
\end{figure}

\section{本文方法}

收缩\&扩展约束(创新点)

\section{实验结果}

\section{讨论}
