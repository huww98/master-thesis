\chapter{基于单张自然环境照片的人脸重建}
\label{chap:recon}

\section{总体目标}

\section{基于神经网络回归的模型初始化}

\section{模型裁剪边缘的梯度处理}
可微分渲染在应用到人脸3D重建时有一个特殊的问题:人是一个很大的物体,而人脸重建的目标通常仅仅是人脸的这一小块区域。
所以通常在人脸重建中用于逆渲染的模型都是裁剪过的,只包含人脸的部分,因此具有一些人为制造的边缘。
这些边缘在照片中并不存在,但本文提出的方法却也会在这些边缘产生相关梯度,导致模型过度扩展,影响最终收敛的效果。对此,本文提出一种简单的基于SDF贴图的方法来区分人为制造的和真正需要优化的边缘。

具体地,本文定义一种SDF贴图。
对于该贴图中的每个像素,其值为一个单通道浮点数$d$,表示该像素点到最近的人为制造边缘的距离,在模型内部的为正,模型外部的为负,
如图\ref{fig:sdf}所示。
nvdiffrast抗锯齿模块工作时将以一对相邻的像素点为单位(如图\ref{fig:aa}),其中一个像素在模型内部,另一个像素在模型外部。
对于处于模型内部的像素,可从SDF贴图中采样其对应的$d$值,并计算$\frac{\partial d}{\partial \mathbf{x}}$,其中$\mathbf{x}$为对应外部像素相对该内部像素的坐标偏移量。
也即对3D模型的表面在该内部像素点的位置使用一个平面进行一阶近似,并求外部像素在该近似平面上的位置,以估计外部像素是否在人为制造边缘之外。
此外,为避免过大梯度是位置估计不准确,本文限制了$\frac{\partial d}{\partial \mathbf{x}}$的最大值。正式地,对于一对分别处于模型内外的像素点,其处于人为制造的边缘当且仅当:
\begin{equation}
    d + \min\left(\frac{\partial d}{\partial \mathbf{x}}, \delta\right) < 0
    \text{,}
\end{equation}
此时应该忽略其产生的可见性梯度。

另一种可能的方案是屏蔽位于模型表面特定区域的像素上的可见性梯度,例如屏蔽特定三角形,到边缘距离小于某个阈值的区域,手动指定遮罩等。
相比于这些方法,本方法具有如下优势:
\begin{enumerate}
\item 渲染尺度无关:同样的模型区域在不同尺度下渲染呈现的尺度不同,因此屏蔽区域的选择可能需要随着尺度变化而改变。
而本方法融入$d$了对像素坐标的梯度信息,因此不受尺度的影响。
\item 有向性:即使是同一个像素,在其不同方向上也可能分别靠近不同种类的边缘。
本方法通过考虑$d$的梯度方向,可以区分这些不同的边缘。
\end{enumerate}

\section{照片到纹理空间信息迁移}

\section{基于可微分渲染的3D重建}

\section{实验结果}

\section{局限性}
