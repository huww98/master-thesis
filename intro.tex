\chapter{绪论}
\label{chap:intro}

\section{研究背景}

3D人脸重建是计算机视觉和计算机图形学领域的一个热门研究方向。
\TODO{元宇宙、游戏、影视}
具有重要现实意义。

可微分渲染
\TODO{参考nvdiffrec论文。背景:渲染用材质复杂化}
\TODO{优势:直接以渲染结果为优化目标,端到端、通用、减少手工步骤。可与传统3D游戏和影视行业的工作流程结合}

当前3D人脸重建的研究方向主要分为两类,包括:
\begin{enumerate}
\item 基于在自然环境中拍摄的少量照片的3D人脸重建方法。
这类方法通常只需要一张在自然环境中使用手机等设备拍摄的照片。
这些方法的目标是在人脸大量先验知识下,重建与输入照片尽可能相似的3D人脸模型。
由于输入数据量的限制,它们通常对物理上几何形状和材质中的细节的准确性要求不高,而是追求主观相似度,以及其作为一个人脸的合理性。
在此方向最近的研究通常是基于深度学习的方法。其中使用可微分渲染技术来辅助神经网络训练的方法也占有一席之地。

\item 基于专用设备拍摄的精细数据的3D人脸重建方法。
这类方法通常基于专业的摄影设备和完全受控的数据处理流程,
通过摄影测量的方法,使用相机作为测量仪器精确定量地测定各个方向上光线强度。
然后根据视差以重建人脸几何形状;根据光源和光线传播的物理规律以估计人脸材质。
这类方法在早期通常采用一些较低精度的数学近似以简化求解,
比如兰伯特(Lambert)反射模型等。
近年来则逐渐追求更加复杂且接近物理规律的模型,
这类方法通常需要较复杂的算法和流程,以及大量的人工工作。
也有不少研究尝试使用可微分渲染来完成端到端地几何和材质的重建。

\end{enumerate}

可见,在3D人脸重建的不同方向的研究中,可微分渲染技术的应用已经成为了一个热门的方向。
应用可微分渲染这一端到端优化方法,可大大降低人工工作量,并提升重建准确度。

\section{本文研究内容及贡献}

本文主要研究可微分渲染技术在3D人脸重建中的理论与应用。
本文从可微分渲染在3D人脸重建中应用时遇到的实际问题出发,
提出了改进的理论,并在实践中验证了其有效性。
并为了发挥数据驱动的算法的优势,为算法收集高质量的数据,搭建了一套高精度的实验平台。
本文的主要贡献可概括如下:

\begin{enumerate}
\item 针对可微分渲染在应用到3D人脸重建时,图像边缘处的梯度无法使3D几何模型与照片中的边缘良好对齐的问题,
本文从理论上分析了其原因是缺乏对背景的建模。
在缺乏对背景的先验知识的前提下,本文提出对可微分渲染的损失函数进行改进,
并通过在收缩、扩张两项梯度项实现了该改进。
并且本文通过玩具实验证明了该项改进可以达成既定的目标,
实现在不对背景建模的情况下,使3D几何模型与照片中的物体边缘良好对齐。

\item 本文实现了一种通过单张在自然环境中拍摄的照片,重建3D人脸模型的方案。
本文在前述理论的基础上,
通过集成神经网络和3DMM人脸模型以提升鲁棒性;
通过可微分渲染技术实现更精准的对齐;
并通过一些传统算法以在3D模型上重现照片中的更多细节。
取各家所长,最终实现了快速高效,细节丰富且全自动的3D人脸模型重建。

\item 本文展示了一套多视角人脸重建实验平台。
其旨在通过软硬件协同设计,在精确的时间点,高分辨率高精度地测定物体反射光线强度数据,
从而为基于物理的可微分渲染的优化提供坚实的基础。
本文介绍了该实验平台的软硬件设计实现,设备标定校准方法和使用交互方式,
并初步验证了其可行性。
\end{enumerate}

\section{本文组织结构}

本文正文共分为六个章节,各章节的主要内容如下:

\newcommand*{\chapref}[1]{\hyperref[{#1}]{第\ref*{#1}章~\nameref*{#1}}}

\chapref{chap:intro}。
本章介绍了3D人脸重建、以及可微分渲染的背景、关联及其研究意义,
并总结了本文主要贡献及各章节的内容安排。

\chapref{chap:related_work}。
本章对3D人脸重建、可微分渲染相关方向的研究和实践进行了综述,
其中包括\TODO{}。
此外,还介绍了一些可用于3D人脸渲染的计算机图形学的基础知识。

\chapref{chap:method}。
本章介绍了本文提出的一种对现有可微分渲染的改进方法。
该方法能增强可微分渲染技术在难以建模背景的图像拟合任务中的适应能力。
其理论构建在修改的损失函数之上,并通过收缩、扩张两项梯度项来实现。
本章也通过实验验证了该方法的有效性。

\chapref{chap:recon}。
利用上一章提出的方法,本章介绍了一个从单张自然环境照片重建3D人脸的实现。
同时该实现有机结合了神经网络、可微分渲染和一些传统算法,
达成了快速高效,细节丰富且全自动的3D人脸模型重建。
本章也与其他相关工作进行了比较。

\chapref{chap:platform}。
为了更专业地采集基于物理的人脸渲染所需的材质数据,实现更高精度,更逼真的3D人脸渲染效果,
本章介绍了一个多视角人脸重建实验平台。
该平台能够实现高分辨率人脸照片的便捷多视角采集,
并为后续的数据收集整理,相机、光源标定等重建所必备的步骤提供了软件支持。
本章将介绍该平台的软硬件设计思路、实现细节、使用方法以及其可行性初步验证的结果。

\chapref{chap:conclusion}。
基于本文完成的工作结果对本文研究进行总结,并分析当前工作存在的不足和对未来工作的展望。
