\chapter{绪论}
\label{chap:intro}

\section{研究背景}

\section{本文研究内容及贡献}

\section{本文组织结构}

本文正文共分为六个章节,各章节的主要内容如下:

\newcommand*{\chapref}[1]{\hyperref[{#1}]{第\ref*{#1}章~\nameref*{#1}}}

\chapref{chap:intro}。
本章介绍了3D人脸重建、以及可微分渲染的背景、关联及意义,
并总结了本文主要贡献及各章节的内容安排。

\chapref{chap:related_work}。
本章对3D人脸重建、可微分渲染相关方向的研究和实践进行了综述,
其中包括\TODO{}。
此外,还介绍了一些可用于3D人脸渲染的计算机图形学的基础知识。

\chapref{chap:method}。
本章介绍了本文提出的一种对现有可微分渲染的改进方法。
该方法能增强可微分渲染技术在难以建模背景的图像拟合任务中的适应能力。
其理论构建在修改的损失函数之上,并通过收缩、扩张两项梯度项来实现。
本章也通过实验验证了该方法的有效性。

\chapref{chap:recon}。
利用上一章提出的方法,本章介绍了一个从单张自然环境照片重建3D人脸的实现。
同时该实现有机结合了神经网络、可微分渲染和一些传统算法,
达成了快速高效,细节丰富且全自动的3D人脸模型重建。
本章也与其他相关工作进行了比较。

\chapref{chap:platform}。
为了更专业地采集基于物理的人脸渲染所需的材质数据,实现更高精度,更逼真的3D人脸渲染效果,
本章介绍了一个多视角人脸重建实验平台。
该平台能够实现高分辨率人脸照片的便捷多视角采集,
并为后续的数据收集整理,相机、光源标定等重建所必备的步骤提供了软件支持。
本章将介绍该平台的软硬件设计思路、实现细节、使用方法以及其可行性初步验证的结果。

\chapref{chap:conclusion}。
基于本文完成的工作结果对本文研究进行总结,并分析当前工作存在的不足和对未来工作的展望。
