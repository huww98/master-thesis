\chapter{相关研究综述}
\label{chap:related_work}

本章主要介绍本课题的相关研究和实践工作,
其中包括高精度的多视角3D重建方法,
基于少量数据的高效3D人脸重建方法,
可微分渲染以及它在人脸重建中的应用。

\section{多视角3D重建}

这类方法力求通过模型来精确描述被采集对象的形状和材质,从而在重新渲染时表现出很高的逼真度。
这些模型通常能从任意角度,在任意光照环境下重新渲染,以满足影视、游戏等领域的需要。

\paragraph{几何形状重建}
几何形状是物体最基本的属性,即物体每个部分在空间中所处的位置。
几何形状重建的基本原理是识别物体同一个部分在不同相机下的匹配关系,然后借助已知的相机参数,以及该部分在不同相机下的像素坐标,计算出该部分在3D空间中的坐标。称为基于视差的方法。

而计算机中描述几何形状的方法有很多种,其中最基本的是点云模型,它保存了大量位于物体表面的点的坐标。
这种表示也是大多数几何重建算法的目标。
最初,\citet{PFM}通过在目标脸上画一些网格,从而允许手动建立格点间的匹配关系,并建立稀疏的点云。
\citet{ss_geo}对多个相机两两之间计算视差图,然后将多组视差图进行融合,并通过约束进一步排除误匹配点,最终得到一个稠密的点云。其中视差图正是两张照片中,像素级别的匹配关系。视差图的构建是分辨率从低到高逐步细化的,并在过程中应用了平滑性、唯一性和顺序性的约束,以消除误匹配点。
这类方法统称为Multi-View Stereo (MVS)。
\citet{DEP}则利用投影仪在目标表面投影彩条图案,以使匹配结果更加鲁棒。
在匹配过程中,镜面反射的效果会随视角的变化而变化,从而导致匹配结果的不准确。
因此有不少方法\citep{DEP}选择使用偏振将镜面反射的效果去除,从而提高匹配的准确度。
但由于次表面散射的存在,这种方法得到的几何形状会过于平滑。但这或许不是一个问题,因为在后续估计材质的过程中,会再估计更高分辨率的法向。
\TODO{light stage}

虽然视差仍然是最广泛使用的方法,其准确度也非常高,但也不乏其他可行的方法。
例如,\citet{MaHPCWD07}使用光度立体的方法,通过多种球面梯度偏振光源的成像效果,重建物体的表面法向。高分辨率的法向信息可与中等分辨率的结构光的结果进行融合,从而获得高分辨率的几何形状,且能避免单纯依靠法线积分带来的累积误差。但其中用到的结构光亦是基于视差的原理。
\citet{phase_shift}则将相移方法与视差结合,以获得更高的准确度和鲁棒性。

在得到点云之后,还需要将点云转换为网格,
该网格的拓扑结构应在采集的不同帧间,以及不同人物间保持一致,
以供渲染、人工编辑,动画制作等使用。
\citet{BeelerHBBBGSG11}\TODO{}

近年来,随着神经辐射场(NeRF)\citep{nerf}等技术的发展,也有一些方法将几何形状编码在神经网络等隐式表达中,而不使用点云、网格等显式表达。
NeRF\citep{nerf}首先提出将几何信息编码在用神经网络表示的密度场中,然后通过体渲染的方法,将渲染结果直接与照片比较,从而优化该模型。
然而,这类隐式模型虽然构建简单,但有诸多问题,例如难以进一步人工编辑,渲染效率较低等。
\citet{nvdiffrec,nvdiffrecmc}先优化在可变形四面体网格中存储的有向距离场,然后将其转换为传统的显式三角形网格,并使用可微分渲染进一步优化。以此解决隐式模型不易编辑的问题。

\paragraph{相机标定}
\citet{ss_geo}采用在一个球形目标物体上随机地粘贴一些可识别的角点。
\citet{del_grid}提出使用三角形网格代替正方形网格,以在交点处提供更多的梯度信息,从而提升角点定位的精度。

\section{高效3D人脸重建}

\section{可微分渲染}

\section{高质量人脸3D渲染}

\paragraph{基于物理的渲染}

\paragraph{分离求和近似}

\paragraph{次表面散射}
